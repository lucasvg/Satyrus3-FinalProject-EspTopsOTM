\documentclass[12pt]{satyrus}
\usepackage{satyrus}
\usepackage[brazil]{babel}

\title{Satyrus III}
\author{Pedro Maciel Xavier}

\begin{document}
    \maketitle
    
    \newpage

    \tableofcontents

    \chapter{Introdução}
    
    \section{Motivação}

    \section{Ficha Técnica}

    \section{Características da Implementação}

    \section{Uso}
    
    \subsection{Instalação}
    
    A instalação pode demandar privilégios de administrador.
    
	\begin{bash}
	~$ git clone
	~$ cd Satyrus3
	~$ sudo python3 setup.py install
	\end{bash}
	
	\begin{shell}
	C:\Users\User> git clone
	C:\Users\User> cd Satyrus3
	C:\Users\User\Satyrus3> python setup.py install
	\end{shell}

    \subsection{Execução}
    
    Escreva seu código em um arquivo de extensão \code{.sat}.
    
	\begin{bash}
	~$ satyrus script.sat
	\end{bash}
	
	\begin{shell}
	C:\Users\User> python -m satyrus script.sat
	\end{shell}

    \chapter{Conceitos Teóricos}

    \section{Lógica Proposicional}
    
    \section{Compiladores}
    
    \chapter{Tipos}
    
    \section{Números}
    
    \section{Matrizes}
    
    \section{Variáveis}
    
    \chapter{Sintaxe do SATish}
    
    \section{Comentários}
    
    \section{Diretivas}
    
    \section{Atribuição}
    
    \section{Matrizes}
    
    \section{Definição de Restrições}
    
    \chapter{Exemplos}
    
    \section{Coloração de Grafos}
    
    \chapter*{Glossário}
    
    \begin{thebibliography}{20}
    	\bibitem{monteiro:10} MONTEIRO, B. F. SATyrus2: \textbf{Compilando Especificações de Racioncínio Lógico}. Dissertação (Engenharia de Sistemas e Computação) - PESC/COPPER, UFRJ. Rio de Janeiro, 2010.
    \end{thebibliography}
    
\end{document}